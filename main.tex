\documentclass{article}
\usepackage{url}

\author{Ethan Mahintorabi, Evan Conrad, Carter Riley}
\date{4/21/2017}
\title{Disabilites and their effect on Transportation Kiosks}

\begin{document}
    \maketitle

    \begin{abstract}
        In our research we found that there are 60,000 people who are legally
        blind acording to a survey done by the city of New York\cite{BLIND}. This
        represents 0.07\% of New York City's population acording to a 2016 estimate of
        the population\cite{CENCUS_ESTIMATE}. In this paper we will dicuss alternate
        user interface types and interactions to help better accomodate the large percentage
        of New Yorkers who have some kind of visual impairment
    \end{abstract}

    \section{Categories of Visual Impairment}
        \paragraph{Partialy Sighted}
                This class of visual impairment is characterized by some kind of vision
            loss resulting in the need for some kind of visual accomodations.\cite{VISAUL_IMPAIRMENT_DEF}
        \paragraph{Low Vision}
            ``Generally refers to a severe visual impairment,
            not necessarily limited to distance vision. Low vision
            applies to all individuals with sight who are unable
            to read the newspaper at a normal viewing distance,
            even with the aid of eyeglasses or contact lenses. They
            use a combination of vision and other senses to learn,
            although they may require adaptations in lighting or
            the size of print, and, sometimes, braille''\cite{VISAUL_IMPAIRMENT_DEF}
        \paragraph{Legally Blind}
            ``Indicates that a person has less than
            20/200 vision in the better eye or a very limited field of
            vision (20 degrees at its widest point)''\cite{VISAUL_IMPAIRMENT_DEF}
        \paragraph{Totally Blind}
            No visual sense information is avalible to a person.\cite{VISAUL_IMPAIRMENT_DEF}

    \section{Challenges for the Visually Impaired}
        The visually imparied have a number of challenges they deal with in their 
        daily lives that effect what they can and cannot do. West et al. performed a study
        to find out to what extent visual impairement has on daily activities. The study found
        specific tasks that become very difficut for those with visual impariment. Some of these
        tasks include plug insertion, key insertion, and dialing a phone\cite{doi:10.1001/archopht.120.6.774}. 
        These tasks are generalized to spatial awareness which appears to be heavliy impacted by visual impairment.
        Subjects in this test were also asked to read printed text which was also heavliy affected by their
        visual impairment.


    \section{Interface Accomodations}
        \subsection{Spatial Identification}
            \paragraph{Kiosk Location} In our research we found that many of that visual impairment affects the 
            ability to place opjects in a spatial enviroment like placing keys in a key hole or in our case
            locating the ticket kiosk. We propose adding a theremin directly under the kiosk machine so that
            support cane users can find the machine by moving their support cane close to the machine. When this
            happens they will hear a different tones as they get closer to the machine.
            

            \paragraph{Payment Location}
                One of the key issues that we identified was that it may difficult for our visually impaired users
                to use the payment system in the kiosks because as Kelly et al. pointed out these users have a 
                difficult time placing objects in an open space\cite{doi:10.1001/archopht.120.6.774}. 
                For example in their paper they pointed at many examples of users having a difficult time placing
                one object in another one. This could present a challenge when trying to place dollar bills in the
                kiosk or a credit card of some kind. \textbf{As a solution to this problem we propose adding the
                same kind of thermin around the location of the payment collection parts of the kiosk.} 

            \paragraph{General Accomodations}
                We plan to add a zooming feature to our interface to help accomadte those who still have some ablity
                to see contrast or large letters. Recent versions of Google Chrome have an ability to smoothly zoom
                in an interface. Adding a read a loud feature is also a requirement for the revised interface.

    
    \section{Revised Use Case}
        \subsection{Main Path}
        \begin{enumerate}
            \item User selects either a ``One Time Pass'' or ``Multi-day pass''
            \item The selects one of the two options.
            \item The system will prompt the user to select a destination in the
            case of the one time pass.
            \item Once a destionation is entered into the system. The system will display
            a total for the user.
            \item The can input cash, credit or debit.
            \item The transaction is completed, and the system will ask if the user
            would like a printout of the directions to their destination.
            \item The user will be given a pass and directions if the user selected it. 
        \end{enumerate}

        \subsection{Alternative Paths}
            \begin{enumerate}
                \item The user places their New York City transport pass on the
                card reader 
                    \begin{enumerate}
                        \item The system will display the amount of credit on the card and the multiday 
                        pass expiration dates. The system will also give the user an option to refill the
                        credit on their card or add a day pass.
                    \end{enumerate}
            \end{enumerate}

    \newpage

    \bibliography{sources} 
    \bibliographystyle{plain}
\end{document}

